% !TEX program = xelatex
\documentclass[12pt, a4paper]{article}

\usepackage{fontspec}
\setmainfont{Pretendard}[
  BoldFont = Pretendard Bold,
  UprightFont = Pretendard,
]
\setmonofont{Consolas}

\usepackage{xcolor}
\definecolor{purple}{HTML}{6C5CE7}
\definecolor{teal}{HTML}{00D2D3}
\definecolor{coral}{HTML}{FF7675}
\definecolor{mint}{HTML}{55EFC4}
\definecolor{dim}{HTML}{888888}
\definecolor{slidebg}{HTML}{F5F5F8}

\usepackage[margin=2cm]{geometry}
\usepackage{enumitem}
\usepackage{tcolorbox}
\tcbuselibrary{skins, breakable}
\usepackage{tabularx}
\usepackage{booktabs}
\usepackage{titlesec}
\usepackage{fancyhdr}
\usepackage{hyperref}
\hypersetup{colorlinks=true, linkcolor=purple, urlcolor=teal}

\pagestyle{fancy}
\fancyhf{}
\renewcommand{\headrulewidth}{0.4pt}
\lhead{\small\textbf{AdStrategy AI} — 1분 발표 스크립트}
\rhead{\small \thepage\ / 3}

\titleformat{\section}{\large\bfseries\color{purple}}{}{0em}{}[\vspace{-0.3em}\color{purple}\rule{\linewidth}{1.5pt}]
\titleformat{\subsection}{\normalsize\bfseries}{}{0em}{}

\setlength{\parindent}{0pt}
\setlength{\parskip}{0.4em}

% 슬라이드 블록
\newtcolorbox{slideblock}[2][]{
  colback=slidebg, colframe=purple!60!black,
  fonttitle=\bfseries, title={#2},
  left=8pt, right=8pt, top=4pt, bottom=4pt,
  boxrule=0.8pt, arc=4pt,
  breakable, #1
}

% 대사 블록
\newtcolorbox{scriptbox}{
  colback=white, colframe=dim!40,
  left=10pt, right=10pt, top=6pt, bottom=6pt,
  boxrule=0.5pt, arc=2pt,
  fontupper=\large,
  breakable
}

% 타이밍 태그
\newcommand{\timing}[1]{\hfill{\small\color{coral}\textbf{#1}}}
\newcommand{\stagenote}[1]{{\small\color{dim}#1}}

\begin{document}

% =====================================================================
\begin{center}
  {\Huge\bfseries AdStrategy AI}\\[0.5em]
  {\Large 1분 발표 스크립트 (인쇄용)}\\[0.3em]
  {\color{dim}\small 총 60초 · 7장 슬라이드 · 데이터 분석 중심 서사}
\end{center}

\vspace{1em}

% =====================================================================
\section{발표 대본 (Slide별)}

% --- Slide 1 ---
\begin{slideblock}{Slide 1 — 가설 \timing{0:00–0:07 (7초)}}
\stagenote{[화면] ``AdStrategy AI'' + 가설 카드 + DATA / MODEL / AGENT 키워드}
\begin{scriptbox}
광고 예산 배분을 데이터로 최적화할 수 있는가.\\
공개 데이터 만 건을 보강하고, 모델과 에이전트까지 연결한 프로젝트입니다.
\end{scriptbox}
\end{slideblock}

% --- Slide 2 ---
\begin{slideblock}{Slide 2 — 파이프라인 \timing{0:07–0:15 (8초)}}
\stagenote{[화면] 6단계 가로 플로우: 데이터 수집 → 보강 → ML → Ablation(빨강) → Agent → 광고주}
\begin{scriptbox}
공개 데이터를 4단계로 만 건까지 보강하고,\\
모델, 감사, 에이전트까지 하나의 파이프라인으로 연결했습니다.\\
\textbf{그런데 ---}
\end{scriptbox}
\end{slideblock}

% --- Slide 3 ---
\begin{slideblock}[colframe=coral!70!black]{Slide 3 — Ablation → Leakage 발견 \timing{0:15–0:25 (10초)}}
\stagenote{[화면] 좌: R²=0.79 빨간 숫자 / 우: 코드 에디터 \texttt{bounce\_rate = 65 - ROAS * 2}}\\
\stagenote{※ 코드 수식은 화면으로만 보여주고, 입으로 읽지 않습니다}
\begin{scriptbox}
예측 모델의 R²가 0.79로 비정상적으로 높았습니다.\\
Ablation Study로 원인을 추적했더니,\\
정답을 암시하는 데이터 누수 변수가 학습에 섞여 있었습니다.
\end{scriptbox}
\end{slideblock}

% --- Slide 4 ---
\begin{slideblock}{Slide 4 — 교정 결과 \timing{0:25–0:33 (8초)}}
\stagenote{[화면] waterfall\_r2\_leakage.png (전체 사용)}
\begin{scriptbox}
해당 변수들을 모두 제거하자 R²는 0.35로 떨어졌습니다.\\
비로소 정답지를 빼고 푼, 모델의 \textbf{`진짜 실력'}을 확인한 겁니다.
\end{scriptbox}
\end{slideblock}

% --- Slide 5 ---
\begin{slideblock}[colframe=teal!70!black]{Slide 5 — 분석→에이전트 연결 + 데모 영상 \timing{0:33–0:47 (14초)}}
\stagenote{[화면] 상단: 미니 파이프라인 (Honest Model → GPT-4o → 4 Tools → Streamlit → 광고주)}\\
\stagenote{슬라이드 2초 노출 후 → Streamlit 녹화 영상으로 전환}
\begin{scriptbox}
이 분석 결과를 AI 에이전트와 연결해,\\
광고주가 대화만으로 데이터 기반 전략을 받을 수 있게 했습니다.
\end{scriptbox}
\stagenote{→ 나머지 ~12초는 데모 영상 (나레이션 없이 자막 또는 무음)}
\end{slideblock}

% --- Slide 6 ---
\begin{slideblock}[colframe=mint!50!black]{Slide 6 — +170\% 임팩트 \timing{0:47–0:53 (6초)}}
\stagenote{[화면] ``+170\%'' 큰 초록 숫자 + 설명 카드 2개}
\begin{scriptbox}
이 정직해진 모델로 상위 캠페인에 예산을 집중 시뮬레이션한 결과,\\
ROAS를 170\%까지 개선할 수 있었습니다.
\end{scriptbox}
\end{slideblock}

% --- Slide 7 ---
\begin{slideblock}{Slide 7 — 핵심 메시지 + 감사 \timing{0:53–1:00 (7초)}}
\stagenote{[화면] 인용문 + ``감사합니다'' + GitHub 로고 + URL}
\begin{scriptbox}
AI의 오류를 잡아내는 건 결국 사람의 몫입니다.\\
정직한 데이터가 진짜 가치를 만듭니다.\\
감사합니다.
\end{scriptbox}
\end{slideblock}

% =====================================================================
\section{타이밍 총정리}

\begin{center}
\begin{tabularx}{\linewidth}{c l c c X}
\toprule
\textbf{\#} & \textbf{슬라이드} & \textbf{시간} & \textbf{초} & \textbf{핵심} \\
\midrule
1 & 가설           & 0:00–0:07 & 7  & 무엇을 검증하려 했나 \\
2 & 파이프라인      & 0:07–0:15 & 8  & E2E 플로우 + ``그런데 ---'' \\
3 & Ablation→Leakage & 0:15–0:25 & 10 & R²=0.79 + 누수 발견 \\
4 & Waterfall       & 0:25–0:33 & 8  & 교정 후 진짜 실력 \\
5 & \textbf{에이전트+데모} & 0:33–0:47 & 14 & 파이프라인 연결 + Streamlit \\
6 & +170\%          & 0:47–0:53 & 6  & 비즈니스 임팩트 \\
7 & 결론+감사       & 0:53–1:00 & 7  & AI는 도구 + 정직한 데이터 \\
\midrule
  & \textbf{합계}   &           & \textbf{60} & \\
\bottomrule
\end{tabularx}
\end{center}

% =====================================================================
\newpage
\section{녹화 가이드라인}

\subsection{준비물 체크리스트}
\begin{itemize}[leftmargin=1.5em]
  \item[$\square$] OBS Studio 설치 (\url{https://obsproject.com})
  \item[$\square$] 마이크 테스트 (내장 마이크 OK, 유선 이어폰 마이크 추천)
  \item[$\square$] Streamlit 앱 미리 열기 (API 키 세팅 완료, ``에이전트 준비 완료'' 확인)
  \item[$\square$] 슬라이드 PPT 발표자 보기(Presenter View)로 열기
  \item[$\square$] 데모 백업 녹화 영상 준비 (동일 시나리오)
  \item[$\square$] 나레이션 3회 연습 (타이머 켜고)
\end{itemize}

\subsection{OBS 설정 (권장)}
\begin{tcolorbox}[colback=slidebg, colframe=dim!40, boxrule=0.5pt, arc=2pt]
\begin{tabularx}{\linewidth}{l X}
\textbf{해상도} & 1920×1080 (FHD) \\
\textbf{FPS} & 30fps (편집 후 최종 출력도 30fps) \\
\textbf{인코더} & x264 또는 NVENC (GPU 있으면) \\
\textbf{비트레이트} & CBR 8,000 kbps (고화질) \\
\textbf{오디오} & 마이크 + 시스템 사운드 분리 녹화 \\
\textbf{포맷} & MKV (녹화 중 크래시 방지) → 완료 후 MP4로 리먹스 \\
\end{tabularx}
\end{tcolorbox}

\subsection{녹화 순서 (2패스 방식)}

\textbf{Pass 1: 슬라이드 + 나레이션}
\begin{enumerate}[leftmargin=1.5em]
  \item OBS 소스: ``화면 캡처'' (PPT 전체 화면) + ``마이크 입력''
  \item PPT를 슬라이드 쇼 모드로 시작
  \item 스크립트를 보면서 Slide 1→7까지 나레이션 녹음
  \item Slide 5에서 ``이 분석 결과를...'' 대사까지 말하고 2초 정지
  \item 완료 후 저장 → \texttt{narration\_raw.mkv}
\end{enumerate}

\textbf{Pass 2: Streamlit 데모 (별도 녹화)}
\begin{enumerate}[leftmargin=1.5em]
  \item OBS 소스: ``화면 캡처'' (브라우저 — Streamlit 앱)
  \item Quick Start → ``핀테크 앱 광고를 미국에서 시작하려고 해요'' 클릭
  \item 에이전트 응답 대기 → 결과 확인 → 예산 시뮬레이터 탭 전환 → 슬라이더 조작
  \item 완료 후 저장 → \texttt{demo\_raw.mkv}
\end{enumerate}

\subsection{편집 (영상 합치기)}

편집 도구: \textbf{CapCut} (무료, 초보 친화) 또는 \textbf{DaVinci Resolve} (무료, 전문가급)

\begin{tcolorbox}[colback=slidebg, colframe=purple!40, boxrule=0.5pt, arc=2pt]
\begin{enumerate}[leftmargin=1.5em]
  \item \texttt{narration\_raw}를 타임라인에 배치
  \item Slide 5 나레이션 끝나는 지점(약 0:35)에서 컷
  \item \texttt{demo\_raw}를 삽입하고 속도 편집:
    \begin{itemize}
      \item 타이핑: 10x 속도
      \item 모델 응답 대기 (스피너): 완전 컷
      \item 응답 생성: 3x 속도
      \item 차트 생성: 원속 (3초)
      \item 탭 전환 + 슬라이더: 5x 속도
      \item 결과 정지: 원속 (3초)
    \end{itemize}
  \item 데모 영상 후 나머지 나레이션(Slide 6–7) 이어 붙이기
  \item 전체 길이 60초 확인 → MP4 1080p 30fps로 내보내기
\end{enumerate}
\end{tcolorbox}

\subsection{비상 대응}
\begin{tabularx}{\linewidth}{l X}
\toprule
\textbf{상황} & \textbf{대응} \\
\midrule
API 응답 지연 / 에러 & 미리 녹화한 데모 백업 영상 사용 \\
나레이션 말 실수 & 해당 슬라이드만 재녹화 후 편집에서 교체 \\
1분 초과 (65초 등) & 슬라이드 전환 사이 무음 구간 잘라내기 (5–10초 절약) \\
1분 초과 (70초+) & 전체 1.05x–1.1x 속도 적용 (목소리 왜곡 최소) \\
\bottomrule
\end{tabularx}

\vfill
\begin{center}
\color{dim}\small
AdStrategy AI — 데이터 기반 광고 전략 설계\\
\url{https://adstrategy-ai.streamlit.app} · \url{https://github.com/Sagak-bit/AdStrategy-AI}
\end{center}

\end{document}
